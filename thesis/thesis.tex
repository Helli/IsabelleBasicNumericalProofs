\documentclass[11pt,a4paper]{article}
\usepackage{isabelle,isabellesym}

% further packages required for unusual symbols (see also
% isabellesym.sty), use only when needed

%\usepackage{amssymb}
  %for \<leadsto>, \<box>, \<diamond>, \<sqsupset>, \<mho>, \<Join>,
  %\<lhd>, \<lesssim>, \<greatersim>, \<lessapprox>, \<greaterapprox>,
  %\<triangleq>, \<yen>, \<lozenge>

%\usepackage{eurosym}
  %for \<euro>

%\usepackage[only,bigsqcap]{stmaryrd}
  %for \<Sqinter>

%\usepackage{eufrak}
  %for \<AA> ... \<ZZ>, \<aa> ... \<zz> (also included in amssymb)

%\usepackage{textcomp}
  %for \<onequarter>, \<onehalf>, \<threequarters>, \<degree>, \<cent>,
  %\<currency>

% this should be the last package used
\usepackage{pdfsetup}

% urls in roman style, theory text in math-similar italics
\urlstyle{rm}
\isabellestyle{it}

% for uniform font size
%\renewcommand{\isastyle}{\isastyleminor}

\newcommand{\snip}[4]
{\expandafter\newcommand\csname #1\endcsname{#4}}
\input{snippets}

\begin{document}

\title{Bachelorarbeit}
\author{Fabian Hellauer}
\maketitle

\tableofcontents

% sane default for proof documents
\parindent 0pt\parskip 0.5ex

\section{Introduction}
The transformation of numerical methods to use machine arithmetic is known to be error-prone. Therefore it is appropriate and important to exercise care to minimize errors. Most of the numerical methods implemented in Isabelle had already been proven to be correct beforehand in Mathematics. While most of these proofs work in a rather general setting, some require additional assumptions or preconditions that are not necessarily given in the context of real operational code, even if it uses the well-known IEEE standard of floating point arithmetic. As a related problem, this standard also introduces a lot of additional cases for operation results with its special values. In consequence, one might consider working with arbitrary precision formats that implement the entire number format using only the infinitely precise integer operations.

% optional bibliography
%\bibliographystyle{abbrv}
%\bibliography{root}

\end{document}

%%% Local Variables:
%%% mode: latex
%%% TeX-master: t
%%% End:
